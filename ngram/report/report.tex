\include{settings}

\begin{document}

\include{title}

\tableofcontents
\newpage

\section{Индивидуальное задание}

\paragraph{Варинат 10.} Определить частоту встречи слов в тексте на русском языке при помощи многопроцессного MPI-приложения. 

\section{Используемое окружение}

\begin{itemize}
	\item ОС: macOS Catalina
	\item Версия ОС: 10.15.4
	\item Процессор: Intel Core i9 @ 2.3GHz × 8
	\item ОЗУ: 16 ГБ
	\item OpenJDK 14
	\item OpenMPI 4.0.3 
\end{itemize}

\section{Алгоритм решения}

Разработку было решено вести на языке Java, с использованием bindings, предоставляемых библиотекой OpenMPI.

Общий интерфейс для поиска частоты вхождения N-грамм можно описать следующим образом:

\begin{lstlisting}
public interface NGramFinder {
	@NotNull
	Map<String, Integer> findNGrams(String[] files) throws Exception;
}
\end{lstlisting}

На вход программе подается набор текстовых файлов, на основе которых и нужно вычислить вероятность появления N-грамм. При этом последовательный алгоритм производит обработку каждого файла последовательно, в то вермя как параллельный алгоритм с использованием OpenMPI разбивает файлы поровну между процессами, уменьшая при этом объем работы одного процесса.

Выходным значением функции является отображение N-граммы на число ее вхождений в переданные текстовые файлы. Для удобства, общее число рассмотренных токенов сохраняется в том же отображении, и используется для вычисления вероятности появления N-граммы $p_{gram}$:
$$
p_{gram} = \dfrac{N_{gram}}{N_{tokens}}
$$

Рассмотрим архитектуру разработанного Java-прилжения. 

\begin{figure}[H]
	\centering
	\includegraphics[width=0.6\linewidth]{classes}
	\caption{Разработанные классы}
\end{figure}

\begin{itemize}
	\item \code{NGram.java} -- класс, содержащий метод \code{main}, который инстаниирует последовательную или паралельную версию определения частот N-грамм.
	\item \code{NGramFinder.java} -- интерфейс, описывающий решаему задчу:
		\begin{itemize}
			\item \code{NGramSeqFinder.java} -- класс, решающий задачу последовательно;
			\item \code{NGramMpiFinder.java} -- класс, использующий интерфейс MPI для обмена инфрмации с другими процессами.
		\end{itemize}
	\item \code{NGramFilter.java} -- интерфейс, описывающий этап фильтрации входного текста:
		\begin{itemize}
			\item \code{NGramCleaner.java} -- класс, удаляющий из входного потока некириллические символы (латиница, пунктуация и т.д.)
			\item \code{NGramTransformer.java} -- класс, применяющий трансформацию к тексу (смена регистра, "схлопывание" пробелов и т.д.)
		\end{itemize}
	\item \code{NGramTokenizer.java} -- класс, разбивающий входной текст на N-граммы.
	\item \code{Utils.java} -- класс, содержащий служебные функции.
\end{itemize}

\noindent Наиболее частотными 3-граммами в рассматриваемых текстах оказались:
\begin{itemize}
	\item что: 0.42\%;
	\item его: 0.37\%;
	\item ост: 0.33\%;
	\item ого: 0.32\%;
	\item про: 0.30\%.
\end{itemize} 

\section{Эксперименты}

В качестве исходных текстов использовались 16 произведений классической русской литературы, такие как ''Война и мир'', ''Анна Каренина'' и др. Для эмуляции большого количества данных, дополнительно были проведены эксперименты, в которых один и тот же файл передавался в качестве входного 10 и 20 раз соответственно.

Число рассматриваемых процессов -- 1 (последовательное выполнение) и от 2 до 16 с шагом в 2 (параллельное выполнение). Для каждого числа процессов измерение времени выполнения было произведено 10 раз, чтобы уменьшить случайность в получаемых данных. 

Для каждого измеренного времени было посчитано ускорение в сравнении с последовательной версией выполнения алгоритма.

\begin{figure}[H]
	\centering
	\includegraphics[width=0.85\linewidth]{all}
	\caption{Время поиска частот N-грамм в зависимости от числа процессов и объема данных}
\end{figure}

\begin{figure}[H]
	\centering
	\begin{subfigure}{0.5\linewidth}
		\includegraphics[width=\linewidth]{1}
		\caption{\code{Data x1}}
	\end{subfigure}
	\begin{subfigure}{0.49\linewidth}
		\includegraphics[width=\linewidth]{10}
		\caption{\code{Data x10}}
	\end{subfigure}
	\begin{subfigure}{0.49\linewidth}
		\includegraphics[width=\linewidth]{20}
		\caption{\code{Data x20}}
	\end{subfigure}
\end{figure}

\begin{figure}[H]
	\centering
	\includegraphics[width=0.85\linewidth]{speedup}
	\caption{Ускорение в зависимости от числа процессов и объема данных}
\end{figure}

Из графиков видно, что при небольшом объеме данных (\code{Data x2}), использование OpenMPI почти не ускоряе выполнение программы, а при числе процессов больше 6 и вовсе замедляет. Время выполнения при этом составляетот 1.5 до 3 секунд.

При среднем (\code{Data x10}) и большом (\code{Data x20}) объеме данных, распараллеливание позволяет значительно ускорить программу (примерно в 3.2 раза), снизив время выполнения с 16 и 32 секунд до 5 и 10 секунд соответстветнно. Оптимальным числом процессов при этом является 6 и 8 соответственно.

Большее ускорение, возможно, не может быть достигнуто ввиду активной работе с дисковой подсистемой в процессе обработки исходных текстов. Данная проблема могла бы быть устранена в случае использования распределенной файловой системы (например, HDFS), и запуска программы на разных узлах MPI-кластера.

\section{Выводы}

В рамках данной лабораторной работы:

\begin{itemize}
	\item разработано Java-приложение, вычисляющее вероятность появления N-грамм в тексте на русском языке;
	\item для распараллеливания исходной задачи использована библиотека OpenMPI;
	\item достигнуто ускорение в 3.2 раза параллельной версии в сравении с последовательной.
\end{itemize}

\end{document}